\documentclass[a5paper,10pt,twoside]{book}

\usepackage[pass]{geometry}
\usepackage{blindtext}
\usepackage{fancyhdr}
\usepackage{graphicx}
\usepackage{titlesec}
\usepackage{wrapfig}
\usepackage[utf8]{inputenc}


% Make it A5.
\pdfpageheight=210mm
\pdfpagewidth=148mm
\addtolength{\textheight}{.875in}

% with this we ensure that the chapter and section
% headings are in lowercase.
\renewcommand{\chaptermark}[1]{\markboth{#1}{}}
\renewcommand{\sectionmark}[1]{\markright{\thesection\ #1}}
\fancyhf{}  % delete current header and footer
\fancyhead[LE,RO]{\bfseries\thepage}
\fancyhead[LO]{\bfseries\rightmark}
\fancyhead[RE]{\bfseries\leftmark}
\renewcommand{\headrulewidth}{0.5pt}
\renewcommand{\footrulewidth}{0pt}
\addtolength{\headheight}{0.5pt} % space for the rule
\fancypagestyle{plain}{%
   \fancyhead{} % get rid of headers on plain pages
   \renewcommand{\headrulewidth}{0pt} % and the line
}

% Remove numbers on empty pages.
\let\origdoublepage\cleardoublepage
\newcommand{\clearemptydoublepage}{%
  \clearpage
  {\pagestyle{empty}\origdoublepage}%
}
\let\cleardoublepage\clearemptydoublepage

\titleformat{\chapter}[display]{}{}{1ex}{\bfseries\LARGE}[]
\titlespacing{\chapter}{0pt}{-100pt}{40pt}

\begin{document}

\begin{titlepage}
\begin{center}
\includegraphics[width=10cm]{boi2014-bw}\\[2cm]
{\bfseries\Huge Baltic Olympiad in Informatics}\\[1.5cm]
{\Huge\it Competition Rules}\\
\vfill
{\bfseries\LARGE Palanga, 2014}
\end{center}
\end{titlepage}

\chapter{Competition Rules}

BOI 2014 is an individual competition between contestants from nine
participating countries: Denmark, Estonia, Finland, Germany, Latvia, Lithuania,
Norway, Poland and Sweden. Each country is allowed to send up to six
contestants.

\section{Competition Schedule}

The competition days of BOI 2014 are Sunday, 27th of April, and Monday, 28th of
April. On each competition day the contestants will have five hours to complete
three tasks.

There will be a Practice Competition held on Saturday, 26th of April, to
familiarise the contestants with the contest environment, including workstations
and the grading system. The solutions submitted during the Practice Competition
will be evaluated, but the results will not be considered in the final ranking.

\section{Environment and Supplies}

For competition each contestant will have a desk with a workstation.

The following software will be available on all workstations:

\begin{enumerate}
    \item Linux Distribution: Debian Wheezy 32-bit, LXDE.
    \item Web browsers: Firefox.
    \item Terminal emulators: lxterminal, xterm.
    \item Editors: joe, vim, gvim, eclipse-cdt, emacs, lazarus, gedit, nano,
    scite, codeblocks, geany, fp.
    \item Compilers and interpreters: gcc/g++ 4.7.1, fpc 2.6.0, python
    2.7.3/3.2.3 debuggers: gdb 7.4.1, ddd 3.3.12.
    \item Documentation for Pascal, C, C++, STL.
\end{enumerate}

Blank, grid paper and pens will be available in the competition room.
Contestants can bring pens, pencils and erasers with them.

If a contestant wishes to bring a keyboard or mouse with wired USB connector or
small mascots or English dictionaries to the competition, these must be
submitted to the Technical Staff during the Practice Competition. Any of these
will be checked and, if cleared, will be provided to the contestant during the
contest. After the first competition day, the contestant must leave these items
on his or her table if (s)he wants to use them during the second competition
day. After the second competition day the contestant must take any of these
items with him or her.

\section{Tasks}

The tasks posed at BOI 2014 are intended to be of algorithmic nature. That is,
the focus is on designing correct and efficient algorithms.

Each task will be divided into one or more subtasks, each worth a portion of the
total points.

\section{Solutions}

Unless otherwise stated in the task description, the solution of a task is a
program written in the Pascal or C++ language. Both C++03 and C++11 versions are
supported. The program submitted by a contestant must be contained in a single
source file.

Solutions will have to fit within time and memory limits specified for every
task separately. The memory limit is on the overall memory usage including
executable code size, stack, heap, etc. Limits are applied to individual test
runs.

Unless otherwise stated in the task description, submissions are required to
read data from standard input and write to standard output.

\section{Starting the Competition}

When contestants enter the competition room at the beginning of the competition,
their workstations will be switched on. Competition tasks will be inside
envelopes next to the workstations. The account details needed to access the
grading system will be given to each contestant in the envelope with the tasks.
Contestants are not allowed to open the envelopes or touch the workstations
until the start signal is given.

Each contestant’s task descriptions and technical information are presented in
English and in his or her native language if such translation is prepared by his
or her team leader. In case of any discrepancies, the English text is binding
and official.

\section{Assistance and Requests}

During the competition communication is allowed only with room supervisors and
the Scientific Committee.

Contestants may submit questions, technical support requests and perform
printing. A contestant may ask a room supervisor for assistance at any time. The
supervisors will NOT answer questions about the competition tasks, but will
deliver paper, printouts, attend to hardware problems, help to find toilets,
etc.

If a contestant encounters problems with hardware, (s)he can request help from
the room supervisor. In case of hardware failure, extra time will be granted to
a contestant but this will cover only the amount of time needed to replace the
hardware.

Questions must be submitted using a special form in the grading system in
English or in the contestant's native language. In the latter case the
Scientific Committee will give answer to the question only after the question is
translated to English by the team leader. Contestants should phrase their
questions so that a yes/no answer would be meaningful. Questions will be
answered with one of the following:

\begin{enumerate}
    \item “YES”
    \item “NO”
    \item “ANSWERED IN TASK DESCRIPTION (EXPLICITLY OR IMPLICITLY)” — The task
    description contains sufficient information. The contestant should read it
    again carefully.
    \item “INVALID QUESTION” — The question is most likely not phrased so that a
    yes/no answer would be meaningful. The contestant is encouraged to rephrase
    the question.
    \item “NO COMMENT” — The contestant is asking for information that the
    Scientific Committee cannot give.
\end{enumerate}

The Scientific Committee will answer every question submitted by contestants.
This may take some time, so the contestant should continue working while waiting
for the answer to a given question. The contestant shall not be involved in the
discussion.

In case of similar questions submitted by several contestants, the Scientific
Committee may give an announcement via the grading system or by presenting it in
competition rooms.

\section{Delivering the Solutions}

Contestants submit their solutions using the grading system and use it to view
the status of their submissions. Each contestant can make at most one submission
for each task per minute at most 100 submissions per task in total.

When a solution is submitted, it will appear in the system, but there may be a
few minutes delay before it is graded. Once it is graded, the contestant will be
able to view summary results of a few sample tests. If these results are
satisfactory, the contestant may release test the submission. This shows the
results of the submission on the official tests for all subtasks. To release
test a submission, it is necessary to forfeit a release token. Contestants will
be given one release token per task every 20 minutes. At any time, at most three
tokens can be saved up for each task.

In the event of technical failure that leads to some contestants not receiving
feedback for their release tested tasks, the rules regarding the allowed
frequency of submissions and release testing may be changed. In such event
students will be notified via the grading system and a live announcement.

Submissions are graded whether or not they are release tested. If a contestant
runs out of tokens, (s)he will not be able to obtain feedback, but can continue
to submit and the last submission will count towards the final grading.

Each input scenario in the grading system will have one of the following
outcomes:

\begin{enumerate}
    \item Correct
    \item Wrong answer
    \item Run-time error (or out of memory)
    \item Time limit exceeded
\end{enumerate}

The grading system will show the running time and memory used by the
contestant’s solution. No information on the actual data or the output produced
by the contestant‘s solution will be given to the contestant.

The contestant may deduce from the feedback his or her provisional score for the
task. However, there is a small chance that the score may change due to appeals,
or indeterminacy of the contestant’s submission. Indeterminacy may arise
intentionally due to the use of pseudo-random number generators, or
unintentionally due to programming bugs or marginal running time. Moreover, in
the event of technical failure or incorrect test data, all submissions may be
re-graded during contest after the problem has been fixed, and this may change
the score. If submissions are re-graded, contestants will be notified via the
grading system and/or a live announcement.

\section{Grading}

Each subtask will be considered solved if every test in it is solved correctly
and within time and memory limits. A submission will receive positive score only
for subtasks that it solves.

The score for each task will be the maximum of any released submissions and the
last submission, whether the last submission is released or not. Submissions may
be re-graded many times, and the final score will be that yielded by the final
grading.

Grading procedures can be overridden in the task description.

\section{Ending the Competition}

Contestants will be given a warning both 30 and 10 minutes before the end of the
competition. After the end of the competition, contestants must immediately stop
working and no further submissions are allowed.

\section{Cheating}

Any of the actions outlined below are considered illegal:

\begin{enumerate}
    \item Using any printed materials, except official BOI 2014 materials or any
    electronic devices or data carriers, except official BOI 2014 equipment.
    \item Bringing to competition room any electronic devices or data carriers.
    \item Using a workstation or an account assigned to someone else.
    \item Communicating in any form to other contestants or people other than
    BOI 2014 staff.
    \item Tampering with or compromising the grading system.
    \item Attempting to store information in any part of the file system other
    than the home directory for their account or the /tmp directory.
    \item Attempting to access any machine on the network or the Internet, other
    than the grading system.
    \item Attempting to reboot or alter the boot sequence of any workstation.
    \item Any other actions that is deemed by the Scientific Committee as
    intentionally aimed at gaining unfair advantage over other contestants.
\end{enumerate}

Moreover, the following rules apply:

\begin{enumerate}
\item Submissions must not attempt to access any files on the file system.
\item Submissions must be single-threaded and must not fork.
\end{enumerate}

Performing illegal actions or breaching any of the rules outlined above may be considered cheating and may result in disqualification.

\section{Appeal Process}

After each competition day every team leader will receive detailed results of
his or her team’s contestants. There will be time allocated to check results on
the grading system.

In case of any disagreement with the results the team leader may submit written
appeal in English. In case of incorrect tests, the incorrect tests are removed
and the scores of all contestants are updated.

Scientific Committee answers each appeal in the written form and gives a short
report to team leaders about all appeals received after each contest day.

\section{Medal Allocation}

After competition all contestants are ranked in descending order with respect to
their final scores. Medals are allocated according to the following algorithm.

\begin{enumerate}
    \item \emph{Score for gold medal} is greatest score where at least 1/12 of
    contestants has final score equivalent or greater than this. All contestants
    whose final score is at least score for gold medal will receive gold medal
    and diploma.
    \item \emph{Score for silver or gold medal} is greatest score where at least
    1/4 of contestants has final score equivalent or greater than this. All
    contestants whose final score is at least score for silver or gold medal and
    who are not awarded by gold medal and diploma (as described above) will
    receive silver medal and diploma.
    \item \emph{Score for medal} is least score where at most 1/2 of contestants
    has final score equivalent or greater than this. All contestants whose final
    score is at least score for medal and who are not awarded by gold or silver
    medal and diploma (as described above) will receive bronze medal and
    diploma.
\end{enumerate}

Only official contestants are counted when calculating the medal boundaries.

\end{document}
